\section{Background}

The root inspiration for this investigation and report was trying to
use Athens, with a validating web-hook, for a company concerned with
what is brought in from external sources.

In the initial setup, there were three intentional (and one
non-intentional) way a package could fail validation. It could have
file(s) that triggered a vulnerability scanner, it could fail to
build, it could have failing unit tests. Or, unintended, either {\tt go
mod download} or {\tt go list -json}\footnote{One possible reason for this is that the GitHub repo has been moved, causing a skew between the downloaded URL and that in the go.mod file} could fail.

It soon became evident that ``has failing unit tests'' was not a
feasible\footnote{Part of this is that over time, multiple ``go vet''
  errors have been promoted test errors} criterion. It eventually
became evident that ``has failing build targets'' was also not
feasible.

This raised a question in the author's mind. What is the current state
of health of the Go eco-system? Previous investigations have answered some of these questions. But, it is interesting to track how this evolves over time.

