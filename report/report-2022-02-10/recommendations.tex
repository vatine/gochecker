\section{Recommendations based on this research}

The conclusions in this section remain mostly unchanged from the first
iteration of this line of investigation.

\subsection{Modules}

By all means make your code into a module, it improves the situation
for any user of your code, and may also make your life easier
(although in some cases it may make it harder).

At least a few of the build errors seen can be directly traced to not
having go modules enabled, so as a general recommendation, it is
probably something that should be done.

\subsection{Renaming}
It is recommended to never rename a Go module. If you want it available
under a new name, consider the new name a ``new module'' and leave
releases up to the point in time where the changed version still
available under the old name. If necessary, archive the repository, so
that no further accidental modifications are possible. Probably after
leaving a link in a README pointing to the new location.

Otherwise, the renaming has suddenly broken previously-fine
packages. If nothing else, the name change is putting a burden on any
user of your library and should ideally be followed up with a pull
request\footnote{The use of github.com as a platform is quite
  prevalent, other version control systems have different names for
  ``this is a unit of change''.} to bring users of the library back to
a building state.

Note that prior to the introduction of the module system, renames were
(to some extent) transparent to code, which is probably why the
practise continues even in a ``we should all be using modules now''
world.

The main reason I am recommending ``don't rename'' is that the name of
a module is (part of) its unique identifier, and with that changing,
on some level it is no longer ``the same module''. If nothing else, it
now has another name.

\subsection{Testing}

It is recommended to have pull requests checked against all (or at
least all relevant) test targets as part of the review
process. Ideally, this should be done by automation, posting status
back to the project.

It is strongly recommended to not cut a release if there are any
failing unit tests.

It may also be useful having periodic (daily? weekly? monthly?) builds
running with the latest toolchain, even if that's not the primary
concern while developing. Sooner or later, your library will end up in
a newer toolchain. And while the Go backwards compatibility guarantees
are pretty good, this report shows that sometimes things change subtly.
